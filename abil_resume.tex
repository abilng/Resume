\documentclass[11pt, a4paper, sans]{moderncv}

\moderncvtheme[green]{classic}

%Alternative commands% \ moderncvtheme:
% \ Moderncvstyle{casual}
% \{Green} moderncvcolor

% Styles: 'casual' (default), 'classic', 'oldstyle' and 'banking'
% Color 'blue' (default), 'orange', 'green', 'red', 'purple', 'gray' and 'black'


% Character encoding
\usepackage[utf8]{inputenc}

% The text width in relation to the width of the paper
\usepackage[scale = 0.8]{geometry}
\recomputelengths

% If you do not want Pagination (one-page document is not numbered)
% \{} nopagenumbers

% Personal Details
% Optional items you may delete or comment
\firstname{Abil}
\familyname{N George}
%\title{Resume}               									% optional
\address{Naithelloor House, Thalachira P.O Kottarakara}{691539 Kollam, Kerala, India}	% optional
\mobile{+91-9497359361}								% optional
\phone{+91-(474)-2402574}                      		% optional
%\fax{+332~(4224)~7834~0134}                        	% optional
\email{mail@abilng.in}                         		% optional
\homepage{www.abilng.in}                   			% optional
%\extrainfo{IIT Roll No:CS13M002}            		% optional
%\photo[100pt]{penguin.jpg}   	% '100pt' is the height of the image is scaled
%\quote{R. Stallman: Snow is so beautiful, it doesn't have to be useful.} % optional


% -----------------------------------------------------------------------------
% content of
% ------------------------------------------------------------------------------
\begin{document}
\maketitle

\section{OBJECTIVE}
To secure a promising, successful and challenging career in a reputed organisation where my knowledge and skill can be effectively applied, enabling me to explore myself fully and realise my full potential.\\


\section{EDUCATION}
%\cventry{year--year}{Degree}{Institution}{City}{ \textit{Grade} }{Description}  % arguments 3 to 6 can be left empty
\cventry{2013-2015}{Master of Technology}{}{}{\hfill \textit{CGPA :8.72}}
{
Indian Institute of Technology Madras, India.\hfill \url{http://www.iitm.ac.in}\\
Concentration: Computer Science \& Engineering.\\
Specialisation: Machine Learning (Deep Neural Networks)\\
}
\cventry{2009-2013}{Bachelor of Technology}{}{}{\hfill \textit{CGPA : 8.64}}
{
University of kerala.\\
College Of Engineering, Trivandrum, Kerala, India.\hfill\url{http://www.cet.ac.in}\\
Concentration: Computer Science \& Engineering.\\
}
\cventry{2009}{All India Senior School Certificate Examination}{}{}{\hfill \textit{Score : 88.80\%}}
{
Central Board of Secondary Education.\\
% Institution :
Jawahar Navodaya Vidyalaya, Pathanamthitta.\hfill \url{http://www.navodaya.nic.in}\\
\textit{JNV are Indian schools for talented students and form a part of the system of gifted education.}\\
}
\cventry{2007}{All India Secondary School Examination}{}{}{\hfill \textit{Score :89.60\%}}
{
Central Board of Secondary Education.\\
Institution : Jawahar Navodaya Vidyalaya, Pathanamthitta. \\
}

\section{TECHNICAL SKILLS}
\cvitem{Programming Languages}{C, C++, Java, Python, Bash, HTML, JavaScript, R, Node.js, \newline \textit{preliminary knowledge: } PHP, Scala, CSS}
\cvitem{Mobile SDK}{Android SDK, iOS \textit{(preliminary knowledge)}}
\cvitem{Operating Systems}{GNU/Linux, Mac OS X, Microsoft Windows}
\cvcomputer{Software Packages}{Eclipse, GCC, GDB, MATLAB, \LaTeX{}}{Databases}{MySQL, Oracle DB, Elastic-Search}

\section{EXPERIENCE}

\cventry{Jun 2016 - }{Software Engineer 2}{PayPal, Bangalore}{}{}
{\textit{Merchant Reporting} - Developed REST API which enable PayPal Customer to access his/her transactional data and derive insights by sharing to third party. Implemented using Java \& Scala. Uses Hadoop (Spark), Kafka \& Elastic Search}

\cventry{Jan 2016 - Jun 2016}{Software Engineer}{PayPal, Bangalore}{}{}
{\textit{Redesign of \textit{PayPal Resolution Center} User interface} - Developed a reusable framework which enables adding new flow within one day by just changing few configurations. Implemented using Node.js, React.js \& Kraken.js}

\cventry{July 2015 - Jan 2016}{Software Engineer}{PayPal, Chennai}{}{}
{\textit{Onboarding API Services} - which provide REST APIs to orchestrate onboarding of new merchants into PayPal \& Braintree ecosystem. Implemented using Java \& spring }


\subsection{Internships}
\cventry{2012-Summer}{Software Development Engineer-Intern}{Amazon.com, Chennai}{}{}
{Implemented an effective framework for automated testing of Kindle Direct Publishing (KDP) Web Interface.}

\cventry{2011-12}{RSMT Algorithm Implementation-Intern}{GES Infotek, Trivandrum }{}{}
{The \textit{Rectilinear Steiner Tree Problem (RSMT)} asks for a minimum length tree that interconnects a given set of points by only horizontal and vertical line segments, enabling the use of extra points. Implemented \textit{ FDP (Fast Dynamic Programming) Algorithm } For RSMT by \textit{ Ganley \& Cohoon } which is based on \textit{ Hwang’s theorem}}.

\section{PROJECTS}
\subsection{Academic Projects}
\cventry{2014-15}{Event Spotting in Video using DNN features}
{\newline \url{https://github.com/abilng/Mtech-Thesis}}
{\hfill \textit{Python, Bash}}
{\newline \textit{Guide: Dr. Hema A. Murthy, Professor, IIT Madras}}
{Images and videos have become ubiquitous on the internet, which has encouraged the development of algorithms for various applications, including search and summarization. Objective is to spot events in videos based on video queries, using DNN features. We have also found a novel method for event recognition in video using Convolutional Neural Networks with pre-processed input.}

\cventry{2014}{Python-DNN - Tool-kit for Deep Neural Network}
{\newline \url{https://github.com/IITM-DONLAB/python-dnn}}
{\hfill \textit{Python, JSON}}
{\newline \textit{Guide: Dr. Hema A. Murthy, Professor, IIT Madras}}
{Python-DNN is a tool-kit for Deep Neural Networks which can run on GPU as well as CPU. It supports CNN, DBN, SDA and many other. \textit{Python-DNN} can be easily configurable by \textit{JSON}. It can be use also as a python library. \newline}

\cventry{2013}{Machine Parsable RESTfull web API}
{\newline \url{https://github.com/abilng/sMash.it}}
{\hfill \textit{JavaScript, Python, Node.js}}
{\newline \textit{Guide: Dr. Abdul Nizar, Professor, College of Engineering Trivandrum}}
{A RESTful web API is a web API implemented using \textit{HTTP }and the principles of \textit{REST (Representational State Transfer)}. By using  a \textit{Microformats}-like grammar that helps to annotate semantics into the already existing documentation of REST services doubling them as machine-readable descriptions. Moreover, these basic annotations help to link between RESTful services in the same domain and enable automatic discovery and composition (creating \textit{Mashups}). \newline}

\iffalse

\cvitem{}{\hspace{3mm}\color{gray}{\textbf{Course Projects}}}

\cventry{2013 \newline Natural Language Processing}
{Sentiment Analysis}{}{}{}
{Sentiment analysis is a supervised learning technique for classifying and/or rating text documents or resources. We make use of a labelled corpus (1000 positive \& 1000 negative cinema reviews) as training set. Used python-NLTK. \newline}

\cventry{2014 \newline Concurrent Programming}
{Static Race Detection And Scalability Analysis Of X10 Programs}{}{}{}
{X10 programming language is an extended subset of Java developed with an aim for
scalable concurrency. Our project is divided into two main parts - \textit{X10 Screen and X10 View}. X10 Screen will identify potential data races in the given parallel input code. While X10 View statically analyse the program's performance and hence analyse its scalability. \newline}

\cventry{2011 \newline Database Design.}
{BOINCr-Graphical Interface to manage BOINC projects}
{\newline \url{https://github.com/dbalan/miniproject}}{}
{\hfill \textit{Java, MySQL}}
{A graphical interface to manage a \textit{BOINC project}. The \textit{Berkeley Open Infrastructure for Network Computing} (BOINC) is an open source middle-ware system for volunteer and grid computing. The \textit{BOINCr} provide a graphical interface to control BOINC server and to allow easy deployment of BOINC apps and work units. \newline}

\fi

\subsection{Other Projects}

\cventry{2012}{ARIA-Ethernet based public announcement system}
{\newline \url{https://github.com/AriaCET/}}
{\newline \textit{Guide: Dr. Rajasree M. S, Professor, College of Engineering Trivandrum}}
{\newline.\hfill \textit{Python (flask), JavaScript, HTML, Bash, Qt (Python)}}
{ARIA \textit{(Asterisk RadIo Architecture)} is an attempt to build a public announcement system over local network which is flexible. ARIA uses VoIP (Voice over IP) and SIP (Session Initiation Protocol). Originally developed for in house use at College of Engineering, Trivandrum (CET). Funded by \textit{Center For Engineering Research And Development (CERD), Govt. Of Kerala}. \newline}

\section{SKILLS \& HOBBIES}
\subsection{Languages}
\cvlistdoubleitem{English \textit{(Professional Proficiency)}}{Malayalam \textit{(Mother tongue)}}
\cvlistdoubleitem{Hindi \textit{(Working Proficiency)}}{Tamil \textit{(Elementary Proficiency)}}

\subsection{Personal Skills}
\cvlistitem{Disciplined, dedicated and hard working with an ability to easily adapt to changing work environment.}
\cvlistitem{Keen learner with ability to learn and imbibe new knowledge.}
\cvlistitem{Able to work independently/ Together in a team.}

\subsection{Hobbies}

\cvlistitem{Contributing to Wikipedia}
\cvlistitem{Listening to Music}
\cvlistitem{Reading}
\cvlistitem{Aquaculture (ornamental-fish)}

\section{PERSONAL DETAILS}
\cvline{}{
\begin{tabular}{ l l }
%\labelitemi~ Name: &Abil N George\\
\labelitemi~ Date of Birth:&05 August 1991\\
%\labelitemi~ Age: &21\\
\labelitemi~ Nationality:&Indian\\
%\labelitemi~ Father name:&George Kutty N A \\
% &\textit{(Ex Servicemen)}\\
%\labelitemi~Mother Name:}&L Leelamma \\
% &\textit{(Teacher-Headmistress)}\\
%\labelitemi~ Marital Status:} &Single\\
\labelitemi~ Permanent Address& \\
&Naithelloor House, \\
&Thalachira P O, \\
&Kottarakara, \\
&Kollam\\
&Pin:691546\\
\labelitemi~ Phone:& \\
&Home: +91(474)2402574\\
&Mobile: +919497359361\\
\labelitemi~ Website: &www.abilng.in\\
\labelitemi~ Email id: &abilngeorge@gmail.com\\
& mail@abilng.in\\
\end{tabular}}

\end{document}
